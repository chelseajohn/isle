\documentclass[a4paper, fleqn, twoside, notitlepage]{scrartcl}
\KOMAoptions{captions=tableheading,
             toc=bibliography
           }

\usepackage[a4paper,
            vdivide={3.3cm,,3.3cm},
            hdivide={2cm,,2cm}
            ]{geometry}

\usepackage{ucs}                % more unicode
\usepackage[utf8x]{inputenc}
\usepackage[T1]{fontenc}
\usepackage[english]{babel}

\usepackage{amsmath}
\usepackage{amsfonts}
\usepackage{amssymb}
\usepackage{amsthm}
\usepackage{mathtools}
\usepackage{commath}            % for nicer differentials
\usepackage{bm}                 % bold math
\usepackage{dsfont}

\usepackage{datetime}
\usepackage{authblk}
\usepackage{picinpar}           % picture in paragraph
\usepackage{graphics}           % addition to above
\usepackage{float}              % place graphics with "H"
\usepackage{caption}
\usepackage{subcaption}
\usepackage{cite}
\usepackage{placeins}           % FloatBarrier
\usepackage[dvipsnames,hyperref]{xcolor}
\usepackage{tcolorbox}
\usepackage[colorinlistoftodos]{todonotes}
\usepackage[pdftex, ocgcolorlinks]{hyperref}
\usepackage{cleveref}

%----------------------------------------------------------
% general info
\newdateformat{isodate}{\THEYEAR-\twodigit{\THEMONTH}-\twodigit{\THEDAY}}

\newcommand{\theauthor}{Jan-Lukas Wynen}
\newcommand{\thetitle}{Algorithms for Fermion action in Hubbard model}
\newcommand{\thedate}{\isodate\today}


\author[1]{\theauthor}
\affil[1]{Institute for Advanced Simulation 4\\
Forschungszentrum J\"ulich, Germany}

\title{\thetitle}
\date{\thedate|\currenttime}

%----------------------------------------------------------
% colors
\definecolor{rwthblau}     {RGB}{0,84,159}
\definecolor{rwthbordeaux} {RGB}{161,16,53}
\definecolor{rwthgelb}     {RGB}{255,237,0}
\definecolor{rwthgrun}     {RGB}{87,171,39}
\definecolor{rwthlila}     {RGB}{122,111,172}
\definecolor{rwthmaigrun}  {RGB}{189,205,0}
\definecolor{rwthmagenta}  {RGB}{227,0,102}
\definecolor{rwthorange}   {RGB}{246,168,0}
\definecolor{rwthpetrol}   {RGB}{0,97,101}
\definecolor{rwthrot}      {RGB}{204,7,30}
\definecolor{rwthturkis}   {RGB}{0,152,161}
\definecolor{rwthviolett}  {RGB}{97,33,88}

\hypersetup{
  colorlinks=true,
  pdftitle={\thetitle},
  pdfauthor={\theauthor},
  linkcolor=rwthblau,
  citecolor=rwthrot,
  urlcolor=rwthturkis,
  linktoc=all   % put links on chapter names and page numbers in toc
}

% colors for highlighting
\colorlet{highl1}{rwthblau}
\colorlet{highl2}{rwthbordeaux}

%----------------------------------------------------------
% color boxes
\newtcolorbox{resultbox}{colback=white, sharp corners, colframe=rwthpetrol}

%----------------------------------------------------------
% abbreviations
\newcommand{\unit}[1]{\,\text{#1}}
\newcommand{\ev}{\,\text{eV}}
\newcommand{\kev}{\,\text{keV}}
\newcommand{\mev}{\,\text{MeV}}
\newcommand{\gev}{\,\text{GeV}}

\let\nodoti\i
\renewcommand{\i}{\mathrm{i}}
\renewcommand{\epsilon}{\varepsilon}

\renewcommand{\Re}{\text{Re}\,}
\renewcommand{\Im}{\text{Im}\,}
\newcommand{\Tr}{\text{Tr}\,}

%----------------------------------------------------------
% miscellaneous
% remove trailing dot from number
\renewcommand*{\figureformat}{%
  \figurename~\thefigure%
}
\renewcommand*{\tableformat}{%
  \tablename~\thetable%
}

% text to be displayed at end of proof
\renewcommand{\qedsymbol}{q.e.d.}

% fallback if hyperref is not included
\providecommand{\url}[1]{\texttt{#1}}

% make "Table" and "Figure" in captions bold
\captionsetup{labelfont=bf}

% bold is more of an eye catcher
% This no longer handles nested emph statements correctly.
\renewcommand{\emph}{\textbf}

% label equations based on section number
\numberwithin{equation}{section}

%%% Local Variables:
%%% mode: latex
%%% TeX-master: "hubbardFermiAction"
%%% End:


\begin{document}

\maketitle
\vspace{-3em}
\tableofcontents

\vfill
\vspace{1em}
\noindent
This document describes the algorithms used for the fermionic action in the Hubbard model.
They are implemented by these classes:
\begin{itemize}
\item \texttt{HubbardFermiMatrix}
\item \texttt{HubbardFermiAction}
\item \texttt{HubbardFermiActionExp}
\end{itemize}
and can be found int the following files:
\begin{itemize}
\item \texttt{src/isle/cpp/action/hubbardFermiMatrix.[hpp|cpp]}
\item \texttt{src/isle/cpp/action/hubbardFermiAction.[hpp|cpp]}
\item \texttt{src/isle/cpp/action/hubbardFermiActionExp.[hpp|cpp]}
\end{itemize}
For usage information, see the source documentation of the classes.\\

\noindent
There are two variants of the algorithms:
\begin{enumerate}
\item Treat particle $M_p$ and hole $M_h$ matrices separately.
\item Use $Q = M_p M_h^T$.
\end{enumerate}
Only Variant 1 is fully implemented.

\clearpage
\section{Variant 1}\label{sec:variant1}

This version treats the fermion matrices for particles and holes separately.
\texttt{HubbardFermiAction} is implemented this way.

\subsection{Definitions}

The fermionic action is
\begin{align}
  S_\text{ferm} = - \log \det M(\phi, \tilde{\kappa}, \tilde{\mu}) M(-\phi, \sigma_{\tilde{\kappa}}\tilde{\kappa}, -\tilde{\mu}) \equiv - \log \det M_p M_h.\label{eq:ferm_action_v1}
\end{align}
where $\phi$ is the auxiliary field that is integrated over. In general $\sigma_{\tilde{\kappa}} = -1$ but for bipartite lattices, a particle-hole transformation can be used to get $\sigma_{\tilde{\kappa}} = +1$. Subscripts $p$ and $h$ indicate matrices for particles and holes, respectively. The fermion matrix is
\begin{align}
  {M(\phi, \tilde{\kappa}, \tilde{\mu})}_{x't';xt}
  &= (1+\tilde{\mu})\delta_{x'x}\delta_{t't} - \tilde{\kappa}_{x'x}\delta_{t't} - \mathcal{B}_{t'} e^{\i\phi_{xt}}\delta_{x'x}\delta_{t'(t+1)}\label{eq:def_m}\\
  &\equiv {K(\tilde{\kappa}, \tilde{\mu})}_{x'x}\delta_{t't} - \mathcal{B}_{t'}{F_{t'}(\phi)}_{x'x}\delta_{t'(t+1)}.
\end{align}
where
\begin{align}
  {K(\tilde{\kappa}, \tilde{\mu})}_{x'x} &= (1+\tilde{\mu})\delta_{x'x} - \tilde{\kappa}_{x'x},\\
  {F_{t'}(\phi)}_{x'x} &= e^{\i\phi_{x(t'-1)}}\delta_{x'x}.
\end{align}
Note that the second $M$ in~\eqref{eq:ferm_action_v1} can not be expressed as $M^*(\phi)$ even for bipartite lattices and $\mu=0$ because $\phi$ is potentially complex valued.
Anti-periodic boundary conditions are encoded by
\begin{align}
  \mathcal{B}_t =
  \begin{cases}
    +1,\quad 0 < t < N_t\\
    -1,\quad t = 0
  \end{cases}
\end{align}
and periodicity in Kronecker deltas.
In time-major layout (time index is slowest) $M$ is a cyclic lower block bidiagonal matrix:
\begin{align}
  M =
  \begin{pmatrix}
    K    &      &        &        & F_0 \\
    -F_1 & K    &        &        &     \\
         & -F_2 & K      &        &     \\
         &      & \ddots & \ddots &     \\
         &      &        &-F_{N_t-1}&K   \\
  \end{pmatrix}.\label{eq:ferm_mat_block_v1}
\end{align}
The parameters are
\begin{itemize}
\item $\delta = \beta / N_t$ and $\beta$ the inverse temperature
\item $\tilde{\kappa} = \delta\kappa$ and $\kappa$ the hopping matrix
\item $\tilde{\mu} = \delta\mu$ and $\mu$ the chemical potential
\end{itemize}


\subsection{LU-Decomposition}\label{sec:lu_v1}

In order to calculate the determinant of $M$ and solve systems of equations, compute the LU-decomposition of $M$. This can be done analytically in terms of spacial matrices given the specific structure in equation~\eqref{eq:ferm_mat_block_v1}.
Use the following ansatz\footnote{This is an adaptation of the algorithm presented in~\cite{zivkovic:2013} and is a simplified version of the algorithm presented in~\cref{sec:lu_decomposition_v2}}:
\begin{align}
  L =
  \begin{pmatrix}
    1   &     &    &        &        &\\
    l_0 & 1   &    &        &        &\\
        & l_1 & \ddots &        &        &\\
        &     & \ddots & 1      &        &\\
        &     &   & l_{n-3} & 1      &\\
        &     &   &        & l_{n-2} & 1
  \end{pmatrix},
  \; U =
  \begin{pmatrix}
    d_0 &     &      &   &        & v_0\\
        & d_1 &     &    &        & v_1\\
        &     & d_2 &    &        & \vdots \\
        &     &     & \ddots &        & v_{n-3} \\
        &     &     &    & d_{n-2} & v_{n-2} \\
        &     &     &    &        & d_{n-1}
  \end{pmatrix}
\end{align}
Note that each $l$, $d$, and $v$ is an $N_x \times N_x$ matrix. Multiplying this out and comparing sides of the equation $M = LU$ leads to a set of recursive equations:
\begin{itemize}
\item $d_i = K$ for $0 \le i \le N_t-2$;\hspace{2em} $d_{N_t-1} = K - l_{N_t-2}v_{N_t-2}$
\item $l_i d_i = -F_{i+1}$ for $0 \le i \le N_t-2$
\item $v_0 = F_0$;\hspace{2em} $l_{i-1} v_{i-1} + v_i = 0$ for $1 \le i \le N_t-2$
\end{itemize}

\noindent
Using $d_i = K$ for $i < N_t-1$, we get
\begin{align}
  l_i = - F_{i+1} K^{-1}\label{eq:mlu_l}
\end{align}
and from this
\begin{align}
  v_i &= -l_{i-1}v_{i-1} = F_i K^{-1} v_{i-1} = F_i K^{-1} F_{i-1} K^{-1} v_{i-2}\\
      &\equiv K A_{0i}\label{eq:mlu_v}
\end{align}
with
\begin{align}
  A_{tt'} \equiv K^{-1} F_{t'} K^{-1} F_{i-1} \cdots K^{-1} F_t.\label{eq:def_partial_A}
\end{align}
Finally, we get the non trivial $d$:
\begin{align}
  d_{N_t-1} = K + F_{N_t-1} K^{-1} K A_{0(n-2)} = K (1 + A),
\end{align}
where $A$ without index is
\begin{align}
  A &\equiv K^{-1}F_{N_t-1}K^{-1} F_{N_t-2} \cdots K^{-1}F_{1}K^{-1} F_{0}.\label{eq:def_A}
\end{align}

\subsection{Action}

Given the LU-decomposition of the previous section, we can write
\begin{align}
  \det M = \det L \det U = (\prod_{i=0}^{N_t-1}\,1) (\prod_{i=0}^{N_t-1}\,d_i)
\end{align}
Resumming ('remultiplying'?) the equations for the non-trivial $d$ gives
\begin{align}
  d_{N_t-1} = K + F_{N_t-1}K^{-1} F_{N_t-2}K^{-1} \cdots F_{1}K^{-1} F_{0}.
\end{align}
Hence the determinant is
\begin{align}
  \det M &= {(\det\,K)}^{N_t-1} \det (K + F_{N_t-1}K^{-1} F_{N_t-2} \cdots K^{-1}F_{1}K^{-1} F_{0})\\
         &= {(\det\,K)}^{N_t} \det(1 + K^{-1}F_{N_t-1}K^{-1} F_{N_t-2} \cdots K^{-1}F_{1}K^{-1} F_{0})
\end{align}
Apply the logarithm to get the final result:
\begin{resultbox}
  \vspace{-\baselineskip}
  \begin{align}
    \log \det M &= N_t \log \det(K)  + \log \det (1 + A),\label{eq:det_M}\\
    A &\equiv K^{-1}F_{N_t-1}K^{-1} F_{N_t-2} \cdots K^{-1}F_{1}K^{-1} F_{0}.
  \end{align}
\end{resultbox}
\noindent
The action~\eqref{eq:ferm_action_v1} is the sum of contributions from particles and holes, i.e.
\begin{align}
  S_\text{ferm} = -N_t \log \det(K_p)  - \log \det (1 + A_p) - N_t \log \det(K_h) - \log \det (1 + A_h).
\end{align}
In general those contributions are not related by a simple equation meaning that both need to be computed from scratch.

\subsection{Force}

The fermionic force at spacetime point $\mu\tau$ is given by
\begin{align}
  {(\dot{\pi}_{\text{ferm}})}_{\mu\tau} = -\dpd{H_{\text{ferm}}}{\phi_{\mu\tau}} = -\dpd{S_\text{ferm}}{\phi_{\mu\tau}} =  \dpd{}{\phi_{\mu\tau}} \Big[\log \det M(\phi, \tilde{\kappa}, \tilde{\mu}) + \log \det M(-\phi, \sigma_{\tilde{\kappa}}\tilde{\kappa}, -\tilde{\mu})\Big].
\end{align}
$\dot{\pi}$ is complex valued in general. For now we will ignore this fact and compute the force from the above equation.
Using~\eqref{eq:det_M} and Jacobi's formula, the derivative can be expressed as
\begin{align}
  {(\dot{\pi}_{\text{ferm}})}_{\mu\tau} = \Tr \Big[{(1+A_p)}^{-1}\dpd{}{\phi_{\mu\tau}}A_p + {(1+A_h)}^{-1}\dpd{}{\phi_{\mu\tau}}A_h\Big].
\end{align}
From now on, $A$ without explicit arguments or subscript refers to $A(\sigma_\phi\phi, \sigma_{\tilde{\kappa}}\tilde{\kappa}, \sigma_{\tilde{\mu}}\tilde{\mu})$ standing for either particle or hole and similarly for $K$ and $F$.\\

\noindent
Using~\cite{henderson:1980} ${(B+C)}^{-1} = C^{-1} - C^{-1}B{(1+C^{-1}B)}^{-1}C^{-1}$ with $B = 1$ and $C = A$, we write
\begin{align}
  \Tr {(1+A)}^{-1}\dpd{}{\phi_{\mu\tau}}A \;=\; \underbrace{\Tr A^{-1}\dpd{}{\phi_{\mu\tau}}A}_{(\text{I})} - \underbrace{\Tr A^{-1}{(1+A^{-1})}^{-1}A^{-1}\dpd{}{\phi_{\mu\tau}}A}_{(\text{II})}.
\end{align}
Calculate the first term. (The expression for $A^{-1}$ is given below in equation~\eqref{eq:def_Ainv})
\begin{align}
  (\text{I}) &= \underbrace{\color{highl1}{(F_0^{-1}K \cdots F_\tau^{-1}K)}_{xa'}}_{(\text{i})} {(F_{\tau+1}^{-1}K)}_{a'a} \underbrace{\color{highl2}{(F_{\tau+2}^{-1}K \cdots F_{N_t-1}^{-1}K)}_{ax'}}_{(\text{ii})}\nonumber\\
             &\quad\times \underbrace{\color{highl2}{(K^{-1}F_{N_t-1} \cdots K^{-1}F_{\tau+2})}_{x'b}}_{(\text{ii})} \Big[\dpd{}{\phi_{\mu\tau}}{(K^{-1}F_{\tau+1})}_{bb'}\Big] \underbrace{\color{highl1}{(K^{-1}F_\tau \cdots K^{-1}F_0)}_{b'x}}_{(\text{i})}\\
             &= {\color{highl1}\delta_{a'b'}^{(\text{i})}}{\color{highl2}\delta_{ab}^{(\text{ii})}} {(F_{\tau+1}^{-1}K)}_{a'a} \i \sigma_{\phi} \delta_{b'\mu} {(K^{-1}F_{\tau+1})}_{bb'}\\
             &= \i\sigma_\phi
\end{align}
Thus the sum of (I) for particles and holes vanishes because ${(\i\sigma_\phi)}_p + {(\i\sigma_\phi)}_h = \i - \i = 0$.
Treat the second term in a similar way:
\begin{align}
  (\text{II}) &= {(1+A^{-1})}^{-1}_{x'y}\; {(F_0^{-1}K \cdots F_\tau^{-1}K)}_{ya'} {(F_{\tau+1}^{-1}K)}_{a'a} \underbrace{\color{highl1}{(F_{\tau+2}^{-1}K \cdots F_{N_t-1}^{-1}K)}_{ay'}}_{(\text{i})}\nonumber\\
              &\quad\times \underbrace{\color{highl1}{(K^{-1}F_{N_t-1} \cdots K^{-1}F_{\tau+2})}_{y'b}}_{(\text{i})} \Big[\dpd{}{\phi_{\mu\tau}}{(K^{-1}F_{\tau+1})}_{bb'}\Big] \underbrace{\color{highl2}{(K^{-1}F_\tau \cdots K^{-1}F_0)}_{b'x}}_{(\text{ii})}\\
              &\quad\times \underbrace{\color{highl2}{(F_0^{-1}K \cdots F_\tau^{-1}K)}_{xc'}}_{(\text{ii})} {(F_{\tau+1}^{-1}K)}_{c'c} {(F_{\tau+2}^{-1}K \cdots F_{N_t-1}^{-1}K)}_{cx'}\nonumber\\
              &= {(1+A^{-1})}^{-1}_{x'y}\; {(F_0^{-1}K \cdots F_\tau^{-1}K)}_{ya'} \underbrace{{(F_{\tau+1}^{-1}K)}_{a'a} {\color{highl1}\delta_{ab}^{(\text{i})}} \i \sigma_\phi \delta_{b'\mu} {(K^{-1}F_{\tau+1})}_{bb'}}_{\i\sigma_\phi\delta_{a'b'}\delta_{b'\mu}}\nonumber\\[-\baselineskip]
              &\quad\times {\color{highl2}\delta_{b'c'}^{(\text{ii})}} {(F_{\tau+1}^{-1}K)}_{c'c} {(F_{\tau+2}^{-1}K \cdots F_{N_t-1}^{-1}K)}_{cx'}\\[1ex]
              &= {(1+A^{-1})}^{-1}_{x'y}\; {(F_0^{-1}K \cdots F_\tau^{-1}K)}_{ya'} \Big[- \dpd{}{\phi_{\mu\tau}}{(F_{\tau+1}^{-1}K)}_{a'c}\Big] {(F_{\tau+2}^{-1}K \cdots F_{N_t-1}^{-1}K)}_{cx'}\\
              &= - \Tr {(1+A^{-1})}^{-1} \dpd{}{\phi_{\mu\tau}}A^{-1}
\end{align}
The minus sign comes from the minus in ${(F^{-1}_{t'})}_{x'x} = e^{-\i\phi_{x(t'-1)}}\delta_{x'x}$ and cancels the minus in front of the whole term.\\
Combining those results gives the momentum as
\begin{resultbox}
  \vspace{-\baselineskip}
  \begin{align}
    {(\dot{\pi}_{\text{ferm}})}_{\mu\tau} &= \Tr \Big[{(1+A_p^{-1})}^{-1}\dpd{}{\phi_{\mu\tau}}A_p^{-1} + {(1+A_h^{-1})}^{-1}\dpd{}{\phi_{\mu\tau}}A_h^{-1}\Big],\label{eq:force_v1}\\
    A^{-1} &\equiv F_0^{-1}K F_1^{-1}K \cdots F_{N_t-1}^{-1}K\label{eq:def_Ainv}
  \end{align}
\end{resultbox}
\noindent Hence it is possible to swap out $A$ for $A^{-1}$ in $\dot{\pi}_{\text{ferm}}$. This improves performance and possibly stability because all constituents ($F^{-1}$ and $K$) are sparse and only inverses of diagonal matrices need to be taken. $A$ on the other side contains $K^{-1}$ which is not trivial to compute and in general dense.\\

\noindent
For reference, write out $\dot{\pi}_{\text{ferm}}$ explicitly. First, in order to improve readability denote partial products of $A^{-1}$ as
\begin{align}
  A^{-1}_{tt'} \equiv F_t^{-1}K \cdots F_{t'}^{-1}K, \qquad A^{-1}_{tt} \equiv F_t^{-1}K
\end{align}
which is the inverse of~\eqref{eq:def_partial_A}.
The general case:
\begin{align}
  {(\dot{\pi}_{\text{ferm}})}_{\mu\tau} = -\i {\big(A^{-1}_{p,(\tau+1)(N_t-1)} {(1+A_p^{-1})}^{-1} A^{-1}_{p,0\tau}\big)}_{\mu\mu} + \i {\big(A^{-1}_{h,(\tau+1)(N_t-1)} {(1+A_h^{-1})}^{-1} A^{-1}_{h,0\tau}\big)}_{\mu\mu}
\end{align}
Elements near the boundary:
\begin{align}
  {(\dot{\pi}_{\text{ferm}})}_{\mu0} &= -\i {\big(A^{-1}_{p,1(N_t-1)} {(1+A_p^{-1})}^{-1} A^{-1}_{p,00}\big)}_{\mu\mu} + \i {\big(A^{-1}_{h,1(N_t-1)} {(1+A_h^{-1})}^{-1} A^{-1}_{h,00}\big)}_{\mu\mu}\\
  {(\dot{\pi}_{\text{ferm}})}_{\mu(N_t-1)} &= -\i {\big(A^{-1}_{p} {(1+A_p^{-1})}^{-1}\big)}_{\mu\mu} + \i {\big(A^{-1}_{h} {(1+A_h^{-1})}^{-1}\big)}_{\mu\mu}\\
  {(\dot{\pi}_{\text{ferm}})}_{\mu(N_t-2)} &= -\i {\big(A^{-1}_{p,(N_t-1)(N_t-1)} {(1+A_p^{-1})}^{-1} A^{-1}_{p,0(N_t-2)}\big)}_{\mu\mu} \nonumber\\
                                 &\quad + \i {\big(A^{-1}_{h,(N_t-1)(N_t-1)} {(1+A_h^{-1})}^{-1} A^{-1}_{h,0(N_t-2)}\big)}_{\mu\mu}
\end{align}
Keep in mind that $\mu$ is \emph{not} summed over, even when it appears twice in one term.

\subsection{Solver}\label{sec:solver_v1}

Linear systems of equations involving $M$ can be solved using an LU-decomposition and forward-/back-substitution.
It is straight-forward to derive expressions for the result using the algorithm presented in Section~\ref{sec:lu_v1}.

\paragraph{Matrix-Vector Equation}
Solve a system of equations for a single right hand side, i.e. $M x = b$, where $x$ and $b$ are vectors.
Start by solving the auxiliary system $L y = b$:
\begin{align}
  y_0 &= b_0\\
  y_i &= b_i + F_i K^{-1}y_{i-1} \quad \text{for} \quad i > 0
\end{align}
And then solve the original equations by solving $U x = y$:
\begin{align}
  x_{N_t-1} &= {(1+A)}^{-1} K^{-1} y_{N_t-1}\\
  x_i &= K^{-1} y_i - A_{0i}x_{N_t-1} \quad \text{for} \quad i < N_t-1,
\end{align}
where the definitions for $A$ and $A_{tt'}$ are given in~\eqref{eq:def_A} and~\eqref{eq:def_partial_A}, respectively.
Note that $y$ always appears as $K^{-1}y$. Hence it is possible to reduce the number of required matrix-vector multiplications by storing $K^{-1}y$ instead of just $y$.


\clearpage
\section{Variant 2}

This version uses the combined matrix $Q = M(\phi, \tilde{\kappa}, \tilde{\mu}) {M(-\phi, \sigma_{\tilde{\kappa}}\tilde{\kappa}, -\tilde{\mu})}^T$ to treat particles and holes together.
Part of this variant is implemented by \texttt{HubbardFermiMatrix} and associated free functions. But a full implementation is not available.

\subsection{Definitions}

The fermionic action can be written as
\begin{align}
  S_\text{ferm} = - \log \det(M(\phi, \tilde{\kappa}, \tilde{\mu})) \det ({M(-\phi, \sigma_{\tilde{\kappa}}\tilde{\kappa}, -\tilde{\mu})}^T) \equiv - \log \det Q(\phi, \tilde{\kappa}, \tilde{\mu}, \sigma_{\tilde{\kappa}}).\label{eq:ferm_action_v2}
\end{align}
Note that the second $M$ can not be expressed as $M^\dagger(\phi)$ even for bipartite lattices and $\mu=0$ because $\phi$ is potentially complex valued. We apply the transpose only to cast $Q$ into a structurally symmetric tridiagonal form. Otherwise it would contain terms $~\delta_{t'(t+2)}$.
Now calculate $Q$ and express it in terms of blocks on the diagonal and off diagonals:
\begin{align}
  {Q(\phi, \tilde{\kappa}, \tilde{\mu}, \sigma_{\tilde{\kappa}})}_{x't',xt}
  &= {M(\phi, \tilde{\kappa}, \tilde{\mu})}_{x't',x''t''} {M^T(-\phi, \sigma_{\tilde{\kappa}}\tilde{\kappa}, -\tilde{\mu})}_{x''t'',xt}\\
  &= \big[(1+\tilde{\mu})\delta_{x'x''}\delta_{t't''} - \tilde{\kappa}_{x'x''}\delta_{t't''} - \mathcal{B}_{t'}e^{\i\phi_{x''t''}}\delta_{x'x''}\delta_{t'(t''+1)}\big] \nonumber\\
  &\quad\times \big[(1-\tilde{\mu})\delta_{x'' x}\delta_{t'' t} - \sigma_{\tilde{\kappa}}\tilde{\kappa}_{x'' x'}\delta_{t'' t} - \mathcal{B}_{t}e^{-\i\phi_{x''t''}}\delta_{x'' x}\delta_{t(t''+1)}\big]\\
  &= \delta_{t't}{(P)}_{x'x} + \delta_{t'(t+1)}{(T^+_{t'})}_{x'x} + \delta_{t(t'+1)}{(T^-_{t'})}_{x'x}
\end{align}
with
\begin{align}
  {P(\phi, \tilde{\kappa}, \tilde{\mu}, \sigma_{\tilde{\kappa}})}_{x'x} &\equiv (2-\tilde{\mu}^2)\delta_{x'x} - (\sigma_{\tilde{\kappa}}(1+\tilde{\mu}) + (1-\tilde{\mu}))\tilde{\kappa}_{x'x} + \sigma_{\tilde{\kappa}}{(\tilde{\kappa}^2)}_{x'x}\\
  {T^+_{t'}(\phi, \tilde{\kappa}, \tilde{\mu}, \sigma_{\tilde{\kappa}})}_{x'x} &\equiv \mathcal{B}_{t'}e^{\i\phi_{x'(t'-1)}}[\sigma_{\tilde{\kappa}}\tilde{\kappa}_{x'x} - (1-\tilde{\mu})\delta_{x'x}]\\
  {T^-_{t'}(\phi, \tilde{\kappa}, \tilde{\mu}, \sigma_{\tilde{\kappa}})}_{x'x} &\equiv \mathcal{B}_{t'+1}e^{-\i\phi_{xt'}}[\tilde{\kappa}_{x'x} - (1+\tilde{\mu})\delta_{x'x}]
\end{align}
and
\begin{align}
  \mathcal{B}_t =
  \begin{cases}
    +1,\quad 0 < t < N_t\\
    -1,\quad t = 0
  \end{cases}
\end{align}
together with periodicity in the Kronecker deltas handles anti-periodic temporal boundary conditions.\\
In time major layout (time is slowest running index) $Q$ assumes a cyclic block tridiagonal form:
\begin{align}
  Q(\phi, \tilde{\kappa}, \tilde{\mu}, \sigma_{\tilde{\kappa}}) =
  \begin{pmatrix}
    P         & T^-_0 &       &         &           &              & T^+_0    \\
    T^+_1     & P     & T^-_1 &         &           &              &          \\
              & T^+_2 & P     & T^-_2   &           &              &          \\
              &       & T^+_3 & P      & \ddots         &              &          \\
              &       &       & \ddots     & \ddots         & T^-_{N_t-3}    &          \\
              &       &       &        & T^+_{N_t-2} & P            & T^-_{N_t-2}\\
    T^-_{N_t-1} &       &       &        &           & T^+_{N_t-1}    & P
  \end{pmatrix}
\end{align}
The parameters are
\begin{itemize}
\item $\delta = \beta / N_t$ and $\beta$ the inverse temperature
\item $\tilde{\kappa} = \delta\kappa$ and $\kappa$ the hopping matrix
\item $\tilde{\mu} = \delta\mu$ and $\mu$ the chemical potential
\end{itemize}

\subsection{LU-Decomposition}\label{sec:lu_decomposition_v2}

The determinant of $Q$ can be computed via an LU-decomposition using an ansatz similar to~\cite{zivkovic:2013}.
\begin{align}
  L =
  \begin{pmatrix}
    1   &     &        &        &        &\\
    l_0 & 1   &        &        &        &\\
        & l_1 & \ddots &        &        &\\
        &     & \ddots & 1      &        &\\
        &     &        & l_{n-3} & 1      & \\
    h_0 & h_1 & \cdots  & h_{n-3} & l_{n-2} & 1
  \end{pmatrix},
  \; U =
  \begin{pmatrix}
    d_0 & u_0 &        &        &        & v_0    \\
        & d_1 & u_1    &        &        & v_1    \\
        &     & d_2    & \ddots &        & \vdots \\
        &     &        & \ddots & u_{n-3} & v_{n-3} \\
        &     &        &        & d_{n-2} & u_{n-2} \\
        &     &        &        &        & d_{n-1}
  \end{pmatrix}
\end{align}
Note that written like this, each component of $L$ and $U$ is an $N_x \times N_x$ matrix, meaning they do not commute.
It is straight forward to derive an iteration procedure to calculate all elements of $L$ and $U$. Most of those relations can be read off immediately, the others can be proven using
simple induction.\\

\noindent
Compute all except last $d, u, l$:
\begin{align}
  \begin{matrix}
    d_0 = P                & u_0 = T^-_0 & l_0 = T^+_1 d_{0}^{-1} & \\
    d_i = P - l_{i-1}u_{i-1} & u_i = T^-_i & l_i = T^+_{i+1} d_{i}^{-1} & \forall i \in [1, N_t-3] \\
    d_{N_t-2} = P - l_{N_t-3} u_{N_t-3} & & &
  \end{matrix}
\end{align}
And all $v, h$:
\begin{align}
  \begin{matrix}
    v_0 = T^+_0           & h_0 = T^-_{N_t-1} d_0^{-1} & \\
    v_i = - l_{i-1} v_{i-1} & h_i = - h_{i-1} u_{i-1} d_{i}^{-1} & \forall i \in [1, N_t-3]
  \end{matrix}
\end{align}
Finally, compute the remaining blocks:
\begin{align}
  u_{N_t-2} &= T^-_{N_t-2} - l_{N_t-3} v_{N_t-3}\\
  l_{N_t-2} &= (T^+_{N_t-1} - h_{N_t-3} u_{N_t-3}) d_{N_t-2}^{-1}\\
  d_{N_t-1} &= P - l_{N_t-2}u_{N_t-2} - \sum_{i=0}^{N_t-3}\, h_i v_i
\end{align}

\noindent
Even though all $P$, $T^+$, and $T^-$ are sparse, the inversions of $d_i$ produce dense matrices in general meaning that this algorithm uses mostly dense algebra.

\subsection{Action}

The action is easy to evaluate once the LU-decomposition of $Q$ is known. It is (see~\eqref{eq:ferm_action_v2})
\begin{resultbox}
  \vspace{-\baselineskip}
  \begin{align}
    S_\text{ferm} = - \log \det Q = -\log \prod_{i=0}^{N_t-1} \det (d_i) = -\sum_{i=0}^{N_t-1}\, \log \det (d_i)
  \end{align}
\end{resultbox}
\noindent
Since the $d$'s are dense, a standard algorithm for computing $\log\det d_i$ can be used.

\subsection{Force}

The fermionic force at spacetime point $\mu\tau$ is given by
\begin{align}
  {(\dot{\pi}_{\text{ferm}})}_{\mu\tau} = -\dpd{H_{\text{ferm}}}{\phi_{\mu\tau}} = -\dpd{S_\text{ferm}}{\phi_{\mu\tau}} =  \dpd{}{\phi_{\mu\tau}} \log \det Q(\phi, \tilde{\kappa}, \tilde{\mu}, \sigma_{\tilde{\kappa}}).
\end{align}
$\dot{\pi}$ is complex valued in general. For now we will ignore this fact and compute the force from the above equation.
Since we have no closed form solution for $\det Q$, we use Jacobi's formula to compute the derivative of $Q$:
\begin{align}
  {(\dot{\pi}_{\text{ferm}})}_{\mu\tau} = \Tr Q^{-1} \dpd{}{\phi_{\mu\tau}} Q
\end{align}
The derivative acts only on $T^\pm$, thus
\begin{align}
  {(\dot{\pi}_{\text{ferm}})}_{\mu\tau}
  &= Q^{-1}_{xt,x't'} \dpd{}{\phi_{\mu\tau}} Q_{x't',xt}\\
  &= \i \big[\delta_{t'(t+1)}\delta_{\tau(t'-1)}\delta_{x'\mu} Q^{-1}_{xt,x't'} {(T^+_{t'})}_{x'x} - \delta_{t(t'+1)}\delta_{\tau t}\delta_{x\mu} Q^{-1}_{xt,x't'} {(T^-_{t'})}_{x'x}\big]
\end{align}
and finally
\begin{resultbox}
  \vspace{-\baselineskip}
  \begin{align}
    {(\dot{\pi}_{\text{ferm}})}_{\mu\tau} = \i \left[{(T^+_{\tau+1})}_{\mu x}Q^{-1}_{x\tau,\mu(\tau+1)} - Q^{-1}_{\mu(\tau+1),x\tau}{(T^-_\tau)}_{x\mu}\right].
  \end{align}
\end{resultbox}
\noindent
Note that $\mu$ and $\tau$ are \emph{not} summed over even though they are repeated on the right hand sides.

\subsection{Solver}

A linear system of equations $Q x = b$ can be solved via an LU-decomposition and forward-/back-substitution.

\paragraph{Matrix-Vector Equation}
Solve a system of equations for a single right hand side, i.e. $x$ and $b$ are vectors. Start by solving the auxiliary system $L y = b$:
\begin{align}
  y_0 &= b_0\\
  y_i &= b_i - l_{i-1} y_{i-1}\quad \text{for}\quad i = 1, \ldots,  N_t-2\\
  y_{N_t-1} &= b_{N_t-1} - l_{N_t-2} y_{N_t-2} - \textstyle\sum_{j=0}^{N_t-3}\, h_j y_j
\end{align}
Then solve $Ux = y$:
\begin{align}
  x_{N_t-1} &= d_{N_t-1}^{-1} y_{N_t-1}\\
  x_{N_t-2} &= d_{N_t-2}^{-1} (y_{N_t-2} - u_{N_t-2} x_{N_t-1})\\
  x_i &= d_i^{-1} (y_i - u_i x_{i+1} - v_i x_{N_t-1}) \quad \text{for} \quad i = 0, \ldots, N_t-3
\end{align}

\paragraph{Inversion}
Invert $Q$ by solving $Q X = \mathds{1}$ for $X \equiv Q^{-1}$, where $\mathds{1}$ is the $N_t N_x \times N_t N_x$ unit matrix.
Start by solving the auxiliary equation $L Y = \mathds{1}$:
\begin{align}
  y_{0j} &= \delta_{0j}\\
  y_{ij} &= \begin{cases}
    \textstyle\prod_{k=j}^{i-1} (-l_{k}) & \mathrm{for}\quad i > j\\
    \delta_{ij} & \mathrm{for}\quad i\leq j
  \end{cases}, \quad\text{for}\quad i = 1, \ldots, N_t-2\\
  y_{(N_t-1)j} &= \delta_{(N_t-1)j} - \textstyle\sum_{k=j}^{N_t-3} h_{k} y_{kj} - l_{N_t-2} y_{(N_t-2)j}
\end{align}
Then solve $U X = Y$:
\begin{align}
  x_{(N_t-1)j} &= d_{N_t-1}^{-1}y_{(N_t-1)j}\\
  x_{(N_t-2)j} &= d_{N_t-2}^{-1}(y_{(N_t-2)j} - u_{N_t-2}x_{(N_t-1)j})\\
  x_{ij} &= d_{i}^{-1}(y_{ij} - u_{i}x_{(i+1)j} - v_{i}x_{(N_t-1)j}) \quad\text{for}\quad i = 0, \ldots, N_t-3
\end{align}
Like in the LU-decomposition itself, those relations can be read off, or proven using simple induction.
Apart from the last row, the $y$'s are independent from each other while the $x$'s have to be computed iterating over rows from $N_t-1$ though 0. However, different columns never mix.

\clearpage
\section{Exponential Hopping}

The fermion action introduced in equations~\eqref{eq:ferm_action_v1} and~\eqref{eq:ferm_action_v2} is an approximation to the `real' Hubbard model action
\begin{align}
  S'_\text{ferm} = - \log \det M'(\phi, \tilde{\kappa}, \tilde{\mu}) M'(-\phi, \sigma_{\tilde{\kappa}}\tilde{\kappa}, -\tilde{\mu}) \equiv - \log \det M'_p M'_h,\label{eq:ferm_action_exp}
\end{align}
with
\begin{align}
  {M'(\phi, \tilde{\kappa}, \tilde{\mu})}_{x't';xt}
  &= (1+\tilde{\mu})\delta_{x'x}\delta_{t't} - \mathcal{B}_{t'} e^{\tilde{\kappa}} e^{\i\phi_{xt}}\delta_{x'x}\delta_{t'(t+1)}\\
  &\equiv (1+\tilde{\mu})\delta_{x'x}\delta_{t't} - \mathcal{B}_{t'}{S(\tilde{\kappa})F_{t'}(\phi)}_{x'x}\delta_{t'(t+1)}.
\end{align}
where
\begin{align}
  {S(\tilde{\kappa})}_{x'x} &= {(e^{\tilde{\kappa}})}_{x'x},\\
  {F_{t'}(\phi)}_{x'x} &= e^{\i\phi_{x(t'-1)}}\delta_{x'x}.
\end{align}
Note that compared to~\eqref{eq:def_m}, the hopping matrix $\tilde{\kappa}$ is on the off-diagonal and in an exponential.
The difference between both actions vanishes in the continuum limit.
However, approximating $e^{\tilde{\kappa}}$ improves performance as $\tilde{\kappa}$ is sparse.
Furthermore, moving the hopping term onto the diagonal improves the behaviour of the Hybrid Monte-Carlo evolution [{\color{rwthrot}Cite ergodicity paper}].\\ %TODO

\noindent
For simplicity, we drop the chemical potential now and \textbf{what follows is only valid for $\mu = 0$!}
The implementation for exponential hopping is also only valid without chemical potential.
This should not be an issue in practice since the algorithm described in Section~\ref{sec:variant1} should be preferred.

\subsection{Action}

The action can be computed easily via the following two substitutions:
\begin{align}
  K &\rightarrow \mathds{1}\label{eq:exp_subs_K}\\
  F_{t'} &\rightarrow S F_{t'}\label{eq:exp_subs_F}
\end{align}
which implies $A \rightarrow B$, see~\eqref{eq:def_B}.
Applied to~\eqref{eq:det_M}, this gives
\begin{resultbox}
  \vspace{-\baselineskip}
  \begin{align}
    \log \det M' &= \log \det (1 + B),\label{eq:det_M_exp}\\
    B &\equiv SF_{N_t-1} S F_{N_t-2} \cdots SF_{1} SF_{0}.\label{eq:def_B}
  \end{align}
\end{resultbox}
And hence the action~\eqref{eq:ferm_action_exp} is
\begin{align}
  S'_\text{ferm} = - \log \det (1 + B_p) - \log \det (1 + B_h).
\end{align}

\subsection{Force}

The force, too, can be calculated by applying substitutions~\eqref{eq:exp_subs_K} and~\eqref{eq:exp_subs_F} to equation~\eqref{eq:force_v1}.
\begin{resultbox}
  \vspace{-\baselineskip}
  \begin{align}
    {(\dot{\pi}'_{\text{ferm}})}_{\mu\tau} &= \Tr \Big[{(1+B_p^{-1})}^{-1}\dpd{}{\phi_{\mu\tau}}B_p^{-1} + {(1+B_h^{-1})}^{-1}\dpd{}{\phi_{\mu\tau}}B_h^{-1}\Big],\\
    B^{-1} &\equiv F_0^{-1}S F_1^{-1}S \cdots F_{N_t-1}^{-1}S
  \end{align}
\end{resultbox}
\noindent
In principle the replacement $B \leftrightarrow B'$ is not useful here because $S$ and $S^{-1}$ are both dense. We apply it nonetheless because it saves us from re-deriving the implementation.

\subsection{Solver}

Again, substitutions~\eqref{eq:exp_subs_K} and~\eqref{eq:exp_subs_F} can be used to transform the solver of Section~\ref{sec:solver_v1} to work for exponential hopping.
The resulting equations are
\begin{align}
  y_0 &= b_0\\
  y_i &= b_i + S F_i y_{i-1} \quad \text{for} \quad i > 0
\end{align}
and
\begin{align}
  x_{N_t-1} &= {(1+B)}^{-1} y_{N_t-1}\\
  x_i &= y_i - B_{0i}x_{N_t-1} \quad \text{for} \quad i < N_t-1.
\end{align}

\clearpage
\bibliographystyle{unsrt}
\bibliography{references}

\end{document}